% The *FIRST* thing you should do is rename the file. While not
% a strict requirement it is recommended that you rename it 
% according to the format T[NUMBER]-theme-name.tex, e.g.
% T1-linear-ds.tex. This makes it easier to spot if you accidently
% for example upload the wrong file.

% The format (A5) is selected to facilitate reading on small
% devices and should NOT be changed. 
\documentclass[a5paper,10pt,oneside]{article}
\AddToHook{cmd/section/before}{\clearpage}
% The package babel is loaded set up for Swedish with Swedish 
% hyphenation,replaces "Contents" with "Innehållsförteckning, 
% "References" with "Litteraturförteckning", etc.
\usepackage[swedish]{babel}

\usepackage[T1]{fontenc}
% The package "inputenc" lets us specify what character encoding
% has been used to save the .tex file. Make sure you set it up
% with the right character encoding, otherwise ÅÄÖ might look 
% wrong, or possibly the document won't compile at all.
\usepackage[utf8]{inputenc}     % Most likely right nowadays, 
                                % might even be standard and not necessary
% \usepackage[latin1]{inputenc} % Possibly right if you use Windows
% Other alternatives are available, but much less likely to be used

% The packages listed below are optional and can be removed if you
% don't use them 
\usepackage{graphicx} 
\usepackage{cite}
\usepackage{url}
\usepackage{ifthen}
\usepackage{listings}	

% These two lines set up options for the listings package and
% can be removed if you don't use it, or changed if you, e.g, 
% use another language than Java. 
% For more information about the listings package see:
% ftp://ftp.tex.ac.uk/tex-archive/macros/latex/contrib/listings/listings.pdf
\lstset{language={},tabsize=3, numbers=left,frame=L,floatplacement=hbtp, basicstyle=\ttfamily, columns=fullflexible}

\usepackage{ifpdf}
\ifpdf
	\usepackage[hidelinks]{hyperref}
\else
	\usepackage{url}
\fi


% Change NR and TITLE below to appropriate values
\title{Tema 2: Algoritmanalys}

% Write the name and user namn for all participants in the group here.
% Separate persons with \and
\author{Wilhelm Durelius \url{widu7139}}

\begin{document}

% Do NOT change the title format in any way, especially not to place it on 
% a separate page
\maketitle

% Here the actual report starts. Everything from here to the start of the
% bibliography should, of course, be removed before you start writing your 
% own text.

\section{Utskrift}
Avgör tidskomplexiteten med big $O$ för pseudokoden nedan. I pseudokoden kan vi komma åt tecken från strängar med array-syntax i $O(1)$-tidskomplexitet.
\lstinputlisting{../examples/example1_constant.pseudo}
\subsection{Svar}
I ett worst-case scenario så körs loopen 5 gånger, resten av koden är $O(1)$. Med tanke på att loopen inte ökar i storlek
när $n$ ($email.length()$) ökar, utan toppar på 5, så har hela metoden en tidskomplexitet på $O(1)$, den är konstant.


\section{Utskrift}
Avgör tidskomplexiteten med big $O$ för pseudokoden nedan. I pseudokoden använder vi StringBuilder för att skriva till strängar i $O(1)$-tidskomplexitet.
Vi kommer även åt tecken i strängar med array-syntax i $O(1)$.
\lstinputlisting[language={}, basicstyle=\ttfamily]{../examples/example2_linear.pseudo}
\subsection{Svar}
I ett worst-case scenario så körs loopen $N$ gånger ($text.length()$), resten av koden är $O(1)$. Med tanke på att loopen ökar i storlek i takt med storleken på $N$,
så är tidskomplexiteten på metoden $O(n)$. 

\section{Utskrift}
Avgör tidskomplexiteten med big $O$ för pseudokoden nedan. Java-versionen av lista.subList(start, end) vi utgår ifrån returnerar i $O(1)$ tidskomplexitet.
\lstinputlisting{../examples/example3_maillog.pseudo}
\subsection{Svar}
I ett worst-case scenario så körs loopen  $log(N)$ gånger ($\frac{N}{2}$ vid varje körning), resten av koden är $O(1)$.
$N$ behöver alltså dubblas för att loopen ska lägga till ett extra varv, vilket gör metoden till $O(\log n)$.

\section{Utskrift}
Avgör tidskomplexiteten med big $O$ för pseudokoden nedan. 
\lstinputlisting{../examples/example4_n2.pseudo}
\subsection{Svar}
I ett worst-case scenario så körs loopen $n* n$ gånger, resten av koden är $O(1)$.
Antalet varv loopen tar i detta fall ökar alltså kvadratiskt i takt med att parametern ökar.
Därför är metodens tidskomplexitet $O(N^2)$.
\end{document}
