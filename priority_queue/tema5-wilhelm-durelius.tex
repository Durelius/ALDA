% The *FIRST* thing you should do is rename the file. While not
% a strict requirement it is recommended that you rename it 
% according to the format T[NUMBER]-theme-name.tex, e.g.

% T1-linear-ds.tex. This makes it easier to spot if you accidently
% for example upload the wrong file.

% The format (A5) is selected to facilitate reading on small
% devices and should NOT be changed. 
\documentclass[a5paper,10pt,oneside]{article}

% The package babel is loaded set up for Swedish with Swedish 
% hyphenation,replaces "Contents" with "Innehållsförteckning, 
% "References" with "Litteraturförteckning", etc.
\usepackage[swedish]{babel}

\usepackage[T1]{fontenc}

% The package "inputenc" lets us specify what character encoding
% has been used to save the .tex file. Make sure you set it up
% with the right character encoding, otherwise ÅÄÖ might look 
% wrong, or possibly the document won't compile at all.
\usepackage[utf8]{inputenc}     % Most likely right nowadays, 
                                % might even be standard and not necessary
% \usepackage[latin1]{inputenc} % Possibly right if you use Windows
% Other alternatives are available, but much less likely to be used

% The packages listed below are optional and can be removed if you
% don't use them 
\usepackage{graphicx} 
\usepackage{cite}
\usepackage{url}
\usepackage{ifthen}
\usepackage{listings}	

% These two lines set up options for the listings package and
% can be removed if you don't use it, or changed if you, e.g, 
% use another language than Java. 
% For more information about the listings package see:
% ftp://ftp.tex.ac.uk/tex-archive/macros/latex/contrib/listings/listings.pdf
\def \lstlistingname {Kodexempel}
\lstset{language=Java,tabsize=3,numbers=left,frame=L,floatplacement=hbtp}


\usepackage{ifpdf}
\ifpdf
	\usepackage[hidelinks]{hyperref}
\else
	\usepackage{url}
\fi



% Change NR and TITLE below to appropriate values
\title{Tema 5: Prioritetsköer}

% Write the name and user namn for all participants in the group here.
% Separate persons with \and
\author{Wilhelm Durelius {widu7139}}

\begin{document}

% Do NOT change the title format in any way, especially not to place it on 
% a separate page
\maketitle

% Here the actual report starts. Everything from here to the start of the
% bibliography should, of course, be removed before you start writing your 
% own text.

\section*{Leftist Heap kontra Binary Heap}
En viktig del i prioritetsköer är funktionaliteten att kunna unionisera dem, att slå ihop två heapar till en.
När man gör en merge i en vanlig DHeap eller Binary Heap tar man helt enkelt värdena på trädet man vill lägga till 
och gör insert på varje värde. Detta tar tid, och är onödigt i och med att trädet man vill slå ihop med redan uppfyller
heap-ordning på sina värden.

För att lösa detta har man designat flera specialiserade varianter av den vanliga array-baserade binära heapen,
en av dem är 'Leftist Heap'. En Leftist Heap använder en länkad datastruktur istället för en array-baserad,
då man kan slå ihop två träd genom att styra om $log(n)$ pekare,
jämfört med att slå ihop två arrays är detta mycket effektivare, $O(log\ n)$ för Leftist Heap kontra $O(n)$ för Binary Heap. 

I binära heapar sker insert av värden i konstant tid, insättning av $n$ värden tar då $O(n)$ tid.
Teoretiskt skulle det kunna ta längre tid, men eftersom värden man stoppar in brukar hamna långt ner i trädet
vilket minskar tidskomplexiteten. Hade man stoppat in ett värde mindre än nuvarande minimum på en binär heap varje gång vid insert hade det blivit $O(log\ n)$
istället för konstant tid.
En Leftist Heap däremot, använder sin Merge-metod för insättningar, men ser det då som ett subträd med 1 nod.
Alla merge-operationer på en Leftist Heap är garanterad $O(log\ n)$ tidskomplexitet. Jämfört med en binär heap
är alltså en enskild insert långsammare på en Leftist Heap, men insert av $n$ värden samtidigt (merge) är snabbare.

Anledningen att en Leftist Heap tar logaritmisk tid är inte enbart på grund av pekar-strukturen, utan 
det som gör en Leftist Heap till en Leftist Heap är Left Heavy NPL-strukturen. NPL (Null Path Length) är längden från en nod till 
den närmaste noden som har mindre än 2 barn. Left Heavy betyder att NPL är större i vänstra subträdet än högra. Leftist Heap byter ut noder horisontellt med varandra vid merge och ser strukturellt till att det vänstra
subträdet har en högre NPL än det högra. Denna 'swap' innebär då tillsammans med pekar-strukturen att ett nodbyte kan flytta ett helt subträd.

RemoveMin i en LeftistHeap använder också merge metoden, 
man tar bort roten och kör merge metoden på de två subträden som blir kvar under.

Sammanfattningsvis, behöver du slå ihop två träd ofta är en Leftist Heap att föredra, men vid behov av enbart enskilda 
insättningar är en binär heap ett bättre alternativ generellt.
\end{document}
